% !TEX root = ../book.tex

\chapter{數論與快速傅立葉轉換}
\label{sec:intro}

本章的作者們都具有數學背景,內容與符號使用上會偏數學一點,因此以下將介紹一些數學背景知識。倘若學員往後要更深入微積分和線性代數,一定可以更加上手!

文中使用\emph{定義}(Definition),以一段簡短的敘述規範一個詞彙或一個概念的意義,建立起讀者與筆者間互相溝通的語言。通常在限定討論內容的範疇時,會使用\emph{集合}指定列舉具有某種性質的總體,集合的呈現有時候會用大寫符號,或是用 $\{\}$ 夾住正在討論的內容。例如 $\mathbb{Z}$ 代表所有正負整數與 $0$,或用 $\{a,b,c,d,e,\ldots,z\}$ 表示所有小寫英文字母。以正式表示法而言,一個元素至多在集合中只能出現一次(不過可以做一些變化讓一個元素出現兩次以上)。在了解集合的定義以後,下文就可以一直使用相同符號代表與定義相同的意義。集合裡面的東西我們稱作\emph{元素},用 $\forall\,x\in\mathcal{S}$ 代表對於所有在集合 $\mathcal{S}$ 中的元素,用 $\exists\,x \in\mathcal{S}$ 代表存在在集合 $\mathcal{S}$ 中的元素是我們要討論的對象。

數學家不喜歡「大概」、「應該」這類的詞彙,當有些性質已經是一定正確(機率 100\%還不保證一定正確喔),我們會用\emph{定理}(Theorem),\emph{引理}(Lemma)來表示一個已經確切知道的現象,通常定理會附上\emph{證明}(Proof)來闡述我們的思路(有時候將證明思路修改之後,就會變成解題的關鍵),不過這裡的證明只會寫上證明思路,不會附上完整證明,而有些定理證明需要的預備工具太多,在此處會略過。

對於一件可以是真是假的事情,我們會說它是一個\emph{性質},性質與性質之間可以有關係,這個關係也是一個性質。例如,(我不出門)是一個性質,(下雨)也是一個性質,(如果下雨,我就不出門)也是一個性質。性質與性質之間可以有等於或不等於的關係,例如,(如果我出門,那一定沒下雨)等於(如果下雨,我就不會出門)。而定理是一個恆為真的性質,$(A)\Rightarrow(B)$ 代表,如果 $(A)$ 成立,則 $(B)$ 也一定會成立。$(A)\Leftrightarrow(B)$ 代表兩者不是同時成立,就是同時不成立。

本章有些演算法會附上程式碼,其難易度並不取決於程式碼長度。程式碼中的命名都不短,目的是希望讀者可以透過檢視其命名就能了解意義,不需再額外註解。此外講者在撰寫本章時,考慮到整體理論表現的完整度,因此將一些測試用的程式碼搭配在一起,讓讀者可以自行比較兩者的差異。


\section{數論 Number Theory}
\label{sec:div}

競賽中數論問題通常會與一些性質結合,本節會講解一些基本的數論定義、定理,以及利用藉由定理基礎寫出競賽中能夠使用的演算法。


\subsection{整除性 Divisibility}
\label{sec:div:div}

\begin{definition}
對於兩個整數 $a,b$,我們說 $a$ \textbf{整除} $b$,或 $b$ 被 $a$ 整除,若存在一個整數 $z$ 使得 $az=b$,記作 $a|b$,此時我們稱 $a$ 是 $b$ 的\textbf{因數},$b$ 是 $a$ 的\textbf{倍數}。
\end{definition}

\begin{definition}
對於一個大於 $1$ 的整數,如果其因數只有自己和 $1$,則稱這個數是\textbf{質數},否則稱為\textbf{合數}。在此定義中 $1$ 既不是質數也不是合數。
\end{definition}

\begin{theorem}
\label{sqrtdivider}
$n$ 是一個合數。存在一個整數 $d$ 使得 $d|n$ 且 $d\leq\sqrt n$
\end{theorem}
\begin{proof}
$n$ 是一個合數代表可以寫為 $n=ab$。
若 $a,b>\sqrt n$,則 $ab>n$。明顯矛盾!
\end{proof}

定理 \ref{sqrtdivider} 提供了一種簡單的質數測試,給定一個大於等於 $2$ 的正整數 $n$,可以測試所有小於等於 $\lfloor\sqrt n\rfloor$ 的整數,是否整除 $n$。若存在 $a$ 使得 $a|n$,則 $n$ 是合數,否則為質數。時間複雜度為 $\mathcal{O}(\sqrt n)$。

\begin{lstlisting}[caption=樸素的質數測試]
bool isPrime(int n) {
  for (int i = 2; i * i <= n; ++i)
    if (n % i == 0) return false;
  return true; 
}
\end{lstlisting}

\begin{definition}
對於一個不知道是否是對的性質 $\mathcal{P}$,使用此符號表示:
$$\left[\mathcal{P}\right]=
\begin{cases}
1 & {\textsf{性質}\mathcal{P}\textsf{是正確的}}\\
0 & {\textsf{性質}\mathcal{P}\textsf{是錯誤的}}
\end{cases}
$$
\end{definition}

\begin{definition}
$$\sum_{d|n} f\left(d\right):=\sum_{x=1}^n \left[x|n\right]f(x)$$
\end{definition}

有一點值得注意的是
$$\sum_{d\vert n} H\left(\frac{n}{d}\right)=\sum_{d\vert n} H(d)$$


\subsection{埃氏篩法 Sieve of Eratosthenes}
\label{sec:div:se}

考慮以下這種題型:
$$\sum_{n=1}^N\sum_{p=1}^n f(p,n)\left[p\textsf{ 是質數}\right][p|n]$$ 限制是 $N\leq 10^5$

廣義而言,我們用 \lstinline{doSomeThing(p, n)} 作為對一些 $p,n$ 進行處理,注意執行順序不能影響答案。在這個例子中是在總和加進 \lstinline{f(p,n)},並假設 \lstinline{isPrime(p)} 可以在常數時間內判斷 $p$ 是不是質數,此時的程式碼會是:

\begin{lstlisting}[caption=硬解]
int ans = 0;
void doSomeThing(p, n) {
  ans += f(p, n);
}
for (int n = 1; n <= N; ++n) {
 for (int p = 1; p <= n; ++p) {
   if (isPrime(p) && n % p == 0) {
     doSomeThing(p, n);
   }
 } 
}
\end{lstlisting}

複雜度分析:$\mathcal{O}(n^2)$,吃了一個大大的TLE饅頭。

注意到此處若將 $\sum$ 交換可得到
$$\sum_{p=1}^N\sum_{n=p}^N f(p,n)\left[p\textsf{ 是質數}\right][p|n]
=\sum_{p=1}^N\left[p \textsf{ 是質數}\right]\left(\sum_{n=p}^N f(p,n)[p| n]\right)$$

括弧內可以運用每 $p$ 個跳一次的方式加速,也就是
$$\sum_{p=1}^N[p\textsf{ 是質數}]\sum_{i=1}^{ip\leq N} f(p,ip)$$

\begin{lstlisting}[caption=交換迴圈的加速效果]
for (int p = 2; p <= N; ++p) {
  if (isPrime(p)) {
    for (int i = 1; (long long) i * p <= N; ++i) {
      doSomeThing(p, i * p);
    }
  }
}
\end{lstlisting}

注意到這兩次 \lstinline{doSomeThing}$(p_i, n_i)$ 中各別 $p_i,n_i$ 的執行次數一樣,只差在執行順序不同,如果有特別的執行順序,可以先丟到某個陣列或是 \lstinline{std::vector} 再排序,最後至多也只有 $\mathcal{O}(N\lg N)$ 個元素。

\problembox{質因數個數估計}{證明題}{
$a$ 是大於一的正整數,令 $f(a)$ 代表 $a$ 的質因數個數,證明 $f(a)\leq\lg a$(這個性質可以幫估在 $N$ 個數字中 $f(a)$ 總和最高只會達到 $\mathcal{O}(n\lg M),a\leq M$)
}

我們可以算算看 \lstinline{doSomeThing} 這個函數被呼叫了幾次:
$$\frac N2+\frac N3+\frac N5+\ldots\frac Np=\mathcal{O}(N\ln\ln N)$$

測出這個上界是因為尤拉老先生證出
$$\sum_{i=1}^N\frac 1{p_i}=\Theta(\ln\ln N)$$

至於要在常數時間求出 \lstinline{isPrime(p)},我們維護 \lstinline{Divider[N]}為$N$的最大質因數。有一種方法可以在常數時間求出 \lstinline{isPrime(p)},先令 \lstinline{Divider} 的每一項都是 $0$,對於每一個 $p$ 以前的質數 $q$,將它的倍數 $n$ 都標記 \lstinline{Divider[n] = q}。當搜尋到 $p$ 時,如果  \lstinline{Divider[p] == 0} 代表小於 $p$ 的質數中前面沒有任何數整除$p$,因此 $p$ 是質數。然後記得將所有 $p$ 的倍數的 \lstinline{Divider} 都標記為 $p$,依此類推,結果程式碼如下:

\begin{lstlisting}[caption=篩法, label=sieveAlgorithm]
int Divider[N + 10] = {};
int main() {
  for (int p = 2 ; p <= N; ++p) {
    if (Divider[p] == 0) { // => p 是質數
      for (int n = p; n <= N; n += p) {
        doSomeThing(p, n) ;
        Divider[n] = p ;
      }
    }
  }
}
\end{lstlisting}
注意到 \lstinline{ Divider[N] } %praser bug
做完以後,對於很多的詢問,每個詢問 $n\leq N$ 可以拿來做很快的($\mathcal{O}(\log n)$)因數分解,或是$\mathcal{O}(1)$測試 $n$ 是不是質數。

\problembox{分解每個詢問}{經典問題}{
給定十萬個小於 $10^6$ 的數字,每個詢問請在 $\mathcal{O}(\lg{10^6})$ 時間內求出質因數分解。
}

如果只是要記任何一個以除以 $n$ 的質因數,用定理 \ref{sqrtdivider} 讓常數小一點,在實作上可以改成:

\begin{lstlisting}[caption=使用篩法測試質因數]
#include <iostream>
typedef long long LL;
const LL N = 1e6;
LL Divider[N + 10] = {};
void sieve() {
  for (LL p = 2; p <= N ; ++p) {
    if (Divider[p] == 0) { // p 是質數
      for (LL n = p * p; n <= (LL) N; n += p) { //差別在此,小心溢位
        // 性質:執行到此時,p | H,H < n
        // 則 Divider[H] != 0(想一下 Divider[H] 會是什麼?)
        Divider[n] = p;
      }
      Divider[p] = p;
    }
  }
}
int main() {
  sieve();
  printf("949327 is %s\n", Divider[949327] == 949327 ?
      "a prime" : "not a prime"
  );
}
\end{lstlisting}

執行程式以前,要不要先猜猜看答案 :-)


\subsection{最大公因數與最小公倍數 GCD \& LCM}
\label{sec:div:gcd}

\begin{definition}
對於整數 $a,b$ 我們稱 $a,b$ 的\textbf{最大公因數},是最大的正整數 $d$ 使得 $d|a$ 且 $d|b$ ,記作 $d=\gcd(a,b)$。當 $\gcd(a,b)=1$ 時,我們稱 $a,b$ \textbf{互質}。
\end{definition}

這是個直觀且好用的定理:

\begin{theorem}
$gcd(a,b)\neq 1\Leftrightarrow\exists\,p$ 是質數使得 $p|a$ 且 $p|b$
\end{theorem}

\problembox{最小字典序}{經典問題}{        
一個陣列有十萬個小於 $10^6$ 的數字。如果兩個相鄰的整數互質則可以交換,否則不行。請問這個陣列的最小的字典序是什麼?        
}


\subsection{輾轉相除法 Euclid Algorithm}
\label{sec:div:eu}

\begin{theorem}
\label{sec:thm:EuclidAlgo}
$\gcd(a,b)=\gcd(b\pmod a,a)$
\end{theorem}

相信大多數人都在國小學過輾轉相除法了,因為用定理 \ref{sec:thm:EuclidAlgo} 遞迴寫就可以把問題慢慢變小,直到一邊整除另一邊馬上返回答案,不過現在要介紹輾轉相除法的擴充算法。

\begin{theorem}
$a,b,r\in\mathbb{Z}$ 若 $d:=\gcd(a,b)$,則找得到 $s,t\in\mathbb{Z}$ 使得 $as+bt=r \Leftrightarrow d|r$
\end{theorem}
\begin{proof}
可以透過輾轉相除法倒推構造 $s,t$ 使得 $as+bt=d$。
\end{proof}

關於如何倒推,我們注意到對於整數 $a,b$ 可以寫成 $r=a-qb$,然後將問題慢慢變小,就請看看下面的程式碼吧!
\begin{lstlisting}[caption=遞迴的擴充輾轉相除法]
#include <utility>
using namespace std;

typedef pair<int , int> ii;
ii exd(int a, int b) { // 回傳 (s, t)
  if (a % b == 0) return ii(0, 1);
  ii T = exd(b, a % b);
  return ii(T.second, T.first - a / b * T.second);
}

int main() {
  int a = 14, b = 8;
  ii ans = exd(a, b);
  printf("gcd(%d, %d) = %d = %d * %d + %d * %d\n", a, b,
    ans.first * a + ans.second * b, ans.first, a, ans.second, b);
}
// gcd(14, 8) = 2 = -1 * 14 + 2 * 8
\end{lstlisting}

\begin{lemma}[歐幾里德引理]
\label{EuclidsLemma}
已知$\gcd(a,b)=1$,那麼若$a|bc$,則$a|c$
\end{lemma}
\begin{proof}
存在 $s,t$ 使得 $as+bt=1$,兩邊同乘以 $c$ 得到 $acs+bct=c\Rightarrow a|c$
\end{proof}


\section{模運算 Modular Arithmetics}
\label{sec:mod}

\subsection{同餘式與模數 Congruence \& Modulo}
\label{sec:mod:mod}

\begin{definition}
若 $n|(a-b)$,則記作 $a\equiv b \pmod n$
\end{definition}
\begin{theorem}
以下是幾個模數運算常見的定理:
\begin{enumerate}
\item $a\equiv b \pmod n,\,c\equiv d \pmod n\Rightarrow a+c \equiv b+d \pmod n$
\item $a\equiv b \pmod n,\,c\equiv d \pmod n\Rightarrow ac \equiv bd \pmod n$
\item $a\equiv b \pmod n\Rightarrow a^k \equiv b^k \pmod n\,\forall k \in\mathbb{N}$
\end{enumerate}
\end{theorem}

這裡挑第三點來證:
\begin{proof}
$$n|(a-b)
\Rightarrow n|(a-b)(a^{k-1}+a^{k-2}b+a^{k-3}b^2+\ldots+ab^{k-2}+b^{k-1})
\Rightarrow n|(a^k-b^k)$$
\end{proof}

利用模數運算,我們可以定義出一種代數結構:
\begin{definition}
對於 $n$ 大於 1,定義 $\langle\mathbb{Z}_n,+_n,\times_n\rangle$,簡寫為 $\mathbb Z_n$,其中:
$$\mathbb Z_n=\{0,1,2,3,4,\ldots,n-1\}$$
$$a+_nb:=(a+b)\pmod n,\,a\times_n b:=(a\times b)\pmod n\,\forall a,b\in\mathbb Z_n$$
\end{definition}

沒錯!這就是平常熟悉的 \%運算,不過請注意正負值。

\begin{theorem}
在 $\mathbb Z_n$ 運算下同樣滿足
\begin{enumerate}
\item 加法乘法結合率:$$(a\times_n b)\times_n c=a\times_n (b \times_n c),\,(a+_n b)+_n c=a+_n (b +_n c)$$
\item 加法乘法交換率:$$a\times_n b=a\times_n b,\,a+_n b=b+_n a$$
\item 分配律:$$(a+_n b)\times_n c=a\times_n c+_n b\times_n c$$
\end{enumerate}
\end{theorem}

有時兩個不同模的數乘以常數後會是同模的,這時有消去定理,同時也給出在模運算之下「除法」的概念:
\begin{theorem}[消去定理]
\label{modcancelationLaw}
$d=\gcd(c,n)$
$$ca\equiv cb \pmod n\Rightarrow a\equiv b \pmod{\frac n d}$$
\end{theorem}
\begin{proof}
因為有定理 \ref{EuclidsLemma},
$$n|c(a-b) \Rightarrow\frac n d\bigg\vert\frac c d (a-b)$$
而且 $$\gcd\left(\frac n d,\frac c d\right)=1$$
我們可以推到$$\frac n d\bigg\vert (a-b)\Rightarrow a\equiv b\pmod{\frac n d}$$
\end{proof}


\subsection{反元素 Inverse Element}
\label{sec:mod:inv}

\begin{definition}
$a,b\in\mathbb Z_n$ 我們稱 $b$ 是 $a$ 的\textbf{乘法反元素}(簡稱反元素),有 $$a\times_n b=1 \Leftrightarrow b:=a^{-1},\,a:=b^{-1}$$
\end{definition}

$\mathbb Z_n$ 中,不是每個數都有反元素。例如,$0$ 在任何 $\mathbb Z_n$ 一定沒有反元素;$\mathbb Z_4$ 中 $2$ 沒有反元素。以 $\mathbb Z_9$ 為例,看看它的反元素是什麼:

\begin{table}[h]
    \begin{tabularx}{\textwidth}{X | X | X | X | X | X | X | X | X | X}
        $a$      & 0 & 1 & 2 & 3 & 4 & 5 & 6 & 7 & 8 \\ \hline
        $a^{-1}$ & 無 & 1 & 5 & 無 & 7 & 2 & 無 & 4 & 8
    \end{tabularx}
    \label{tab:mod:inv:table}
    \caption{$\mathbb Z_9$ 乘法反元素表}
\end{table}

由此可以觀察可發現以下定理:
\begin{theorem}
$a$ 在 $\mathbb Z_n$ 中有乘法反元素若且唯若 $\gcd(a,n)=1$。
\end{theorem}
\begin{proof}
輾轉相除擴充演算法總是能把反元素造出來,可以利用擴充算法找模 $n$ 下的反元素:$$ax+py=1\Leftrightarrow ax=1-py\Leftrightarrow ax \equiv 1 \pmod p\Leftrightarrow x\equiv a^{-1}\pmod p$$
\end{proof}
輾轉相除擴充演算法分解的步驟很少,寫成迭代更快!

\begin{theorem}
\label{UniqueInverse}
反元素是唯一的,也就是說如果 $a\times_n b=1,\,a\times_n c=1$ 則 $b\equiv c\pmod n$
\end{theorem}
\begin{proof}
$a\times_n (b-c)=0$,也就是 $n|a(b-c)$ 。\\
因為 $\gcd(n,a)=1$,利用定理 \ref{EuclidsLemma} 可以推得 $n|(b-c)$
\end{proof}


\subsection{費馬小定理 Fermat's Little Theorem}
\label{sec:mod:FLT}
\begin{theorem}[費馬小定理]
對於任何整數 $a$,質數 $p$ 皆滿足 $a^p-a\equiv 0\pmod p$。若 $a$ 不是 $p$ 的倍數,則可以推得 $a^{p-1} \equiv 1\pmod p$
\end{theorem}
\begin{proof}
費馬有一個很關鍵的發現:
對於任何 $a\in \mathbb Z_n,\,a\neq 0$,在
$$1a,2a,3a,4a,\ldots,(p-1)a $$ 中必須一一對應到 $$1,2,3,4,\ldots,(p-1)$$
否則,會造成矛盾:$i>j,\,ia\equiv ja\pmod p\Rightarrow p|(i-j)a$。
$$1a,2a,3a,4a,\ldots,(p-1)a ,\ 1,2,3,4,\ldots,(p-1)$$
在兩邊連乘得到:
$$(p-1)!a^{p-1}=a\cdot 2a \cdot 3a\cdots (p-1)a\equiv 1\cdot 2\cdot 3\cdots (p-1)=(p-1)! \pmod p$$
根據定理 \ref{modcancelationLaw},兩邊消去 $(p-1)!$ 得到 $a^{p-1}\equiv 1\pmod p$\\
這說明了本節的重點:任何一個非 $p$ 倍數的正整數,都有乘法反元素 $a^{p-2}$(還記得定理 \ref{UniqueInverse} 說明反元素是唯一的嗎?)
\end{proof}

現在我們知道 $\mathbb Z_p$ 的元素支援加減,非零元素支援乘除。像這樣的性質的結構,數學家們稱作是一個\textbf{體}(Field)。

因此可以使用快速冪來求 $a^{p-2}$:

\begin{lstlisting}[caption=快速冪求反元素]
typedef long long LL;
LL pow(LL a, LL p, LL mod) { // 也可以寫寫看遞迴版本
  if (a % mod == 0) return 0;
  LL ret = 1ll % mod;
  for (LL cur = a; p; cur = cur * cur % mod, p >>= 1) {
    if (p & 1ll) {
      ret = ret * cur % mod;
    }
  }
  return ret;
}
int main() {
  const LL prime = 7;
  for (LL a = 1, inva; a < prime; ++a) {
    inva = pow(a, prime - 2, prime);
    printf("a = %2lld, a^-1 = a^(p-2) = %2lld, a * a^(p-2) = %lld\n",
        a, inva, a * inva % prime);
  }
}

/*
a =  1, a^-1 = a^(p-2) =  1, a * a^(p-2) = 1
a =  2, a^-1 = a^(p-2) =  4, a * a^(p-2) = 1
a =  3, a^-1 = a^(p-2) =  5, a * a^(p-2) = 1
a =  4, a^-1 = a^(p-2) =  2, a * a^(p-2) = 1
a =  5, a^-1 = a^(p-2) =  3, a * a^(p-2) = 1
a =  6, a^-1 = a^(p-2) =  6, a * a^(p-2) = 1
*/
\end{lstlisting}


\subsection{歐拉函數 Euler Function}
\label{sec:mod:eu}

\begin{definition}
\textbf{歐拉函數} $\Phi(n)$ 的值表示在 $1,2,\ldots,n$ 中與 $n$ 互質的個數。
\end{definition}

我們使用 $\mathbb Z_n^*$ 代表從 $[0,n)$ 中與 $n$ 互質的數,也可以說 $\mathbb Z_n^*$ 有 $\Phi(n)$ 個元素。

\begin{theorem}[Euler]
如果 $\gcd(a,n)=1$, 則 $a^{\Phi(n)}\equiv 1\pmod n$
\end{theorem}
\begin{proof}
注意到費馬小定理的證法,如果 $\gcd(a,n)=1$,也可以用來證明 $a^{\Phi(n)}\equiv 1\pmod n$
\end{proof}

\begin{theorem}
若 $\gcd(m,n)=1$,則 $\Phi(mn)=\Phi(m)\Phi(n)$
\end{theorem}
\begin{proof}
我們可以注意下圖,可以發現只有 $\Phi(m)$ 個列有跟 $m$ 互質的數(為什麼呢?),每個列剛好有 $\Phi(n)$ 個項跟 $n$ 互質(因為列中的元素會跟 $0,1,2,\ldots,n-1\pmod n$ 一一對應)
\end{proof}
$$
\begin{array}{ccccc}		
   1 & 2 & 3 & \dots & m \\		
   m+1 & m+2 & m+3 & \dots & 2m \\	
   \vdots & \vdots & \vdots & \vdots & \vdots \\
   (n-1)m+1 & (n-1)m+2 & (n-1)m+2 & \dots & nm 	
\end{array}
$$

\begin{theorem}
$$\sum_{d|n}\Phi(d)=n$$
\end{theorem}
\begin{proof}
若 $d|n$,在 $\{1,2,3,4,5,\ldots,n\}$ 裡面總共有 $\Phi(d)$ 個元素跟 $n$ 的最大公因數是$\frac n d$
則 $\sum_{\frac n d|n} | \{a|gcd(a,n)=\frac n d\}|=\sum_{\frac n d |n}\Phi(d)=\sum_{d|n}\Phi(d)=n$
\end{proof}

請從上面性質自行推出歐拉函數公式:
\problembox{歐拉函數公式}{證明題}{
給予一個$n=p_1^{k_1}p_2^{k_2}p_3^{k_3}\cdots p_n^{k_n}$,請算出$\Phi(n)$。
}


\subsection{數論函數 Number-Theoretic function}
\label{sec:mod:NTF}

\begin{definition}
一個數論函數 $f(n)$ 是一個定義在正整數的函數,值域是實數(或複數)。
\end{definition}
簡單來說數論函數吃一個正整數,吐出一個實數或複數。
\begin{definition}
設數論函數 $f(n)$,若 $\gcd(m,n)=1$,滿足 $f(1)=1$ 且 $f(mn)=f(m)f(n)$ 的話我們稱 $f$ 為\textbf{積性函數}(Multiplicative function)
\end{definition}
例子像是 $\Phi(x)$ 是積性函數,考慮到任何一個正整數可以分解成質數的冪次,要算積性函數$f(p_1^{k_1}p_2^{k_2}...p_n^{k_n})$,我們只要計算$f(p_1^{k_1})f(p_2^{k_2})...f(p_n^{k_n})$ 即可

\problembox{構造積性函數}{經典問題}{
給一堆兩兩互質的正整數 $\{a_i\}_{i=0}^{N-1},a_i\leq 10^6$,還有 $f(a_i)$ 們也給了,請造出任何一個符合條件積性函數 $f$,構造方法為輸出 $f(1),f(2),\ldots,f(10^6)$,若無解請輸出 $-1$
}

\begin{theorem}
若 $f(n)$ 是積性函數,則 $F(n)=\sum_{d|n} f(d)$ 也是積性函數
\end{theorem}
\begin{proof}
若 $m,n$ 互質,可以把任何 $mn$ 的因數拆開寫成 $d_1|m,d_2|n$,$d_1d_2|mn$
$$F(mn)=\sum_{d|nm} f(d)=\sum_{d_1|m,d_2|n}f(d_1)f(d_2)=\sum_{d_1|m}f(d_1)\sum_{d_2|n}f(d_2)=F(m)F(n)$$
\end{proof}

對於數論函數,有所謂狄利克雷卷積(Dirichlet convolution):
\begin{definition}
我們定義兩個數論函數 $f,g$ 的\textbf{狄利克雷卷積}為
$$(f*g)(n):=\sum_{d_1|n} f(d_1)g(\frac n {d_1})$$
或者寫得更直觀一點:
$$(f*g)(n):=\sum_{d_1d_2=n} f(d_1)g(d_2)$$
三個函數的狄利克雷卷積長這樣,依此類推
$$((f* g)* h)(n)=\sum_{d_1d_2d_3=n} f(d_1)g(d_2)h(d_3)$$
\end{definition}

很明顯的可以看出狄利克雷卷積有結合率、交換率:
\begin{enumerate}
\item $(f* g)* h \\ =\sum_{d_1d_2d_3=n} f(d_1)g(d_2)h(d_3)=f* (g*h)$
\item $f* g \\ =\sum_{d_1d_2=n} f(d_1)g(d_2)=g* f$
\end{enumerate}

以下介紹幾個等等會用到的積性函數,可以自行驗證看看它是否是積性函數

\begin{enumerate}
\item $I(x):=[x=1]$
\item 常數函數1:$1(x):=1$
\item 莫比烏斯函數:
$$\mu(x)=
\begin{cases}
1 & x=1\\
0 & p^{2}\vert x,\,p\textsf{ 是質數}\\
(-1)^k & x=p_1p_2\cdots p_k
\end{cases}
$$
\end{enumerate}
注意一下 $$\sum_{d\vert x} \mu(d)=[x=1]=I(x)$$(可令 $x=p^k$ 證證看)


\subsection{莫比烏斯反演 Möbius Inversion Formula}
\label{sec:mod:MIF}

\begin{theorem}
對於數論函數 $f$ 跟 $F$ 關係如下:
$$F\left(x\right)=\sum_{d\vert x} f\left( d\right)$$
那麼我們有:
$$f\left(x\right)=\sum_{d\vert x} \mu\left(d\right)F \left(\frac x d  \right)$$
\end{theorem}
\begin{proof}
核心的證明想法就是交換 $\sum$ 的技巧,這在比賽中很常見
$$\sum_{d\vert x} \mu(d)F \left(\frac x d  \right)=\sum_{d\vert x} \mu(d)\sum_{c|\frac x d} f(c)=\sum_{d\vert x}\sum_{c\ | \frac x d}f(c)\mu(d)$$
改成這個寫法好看多了
$$\sum_{cd|x} f(c)\mu(d)$$
拜託拜託,一定要注意到\\
$$(d|x\textsf{ 且 }c|\frac x d)\Leftrightarrow cd|x\Leftrightarrow (c|x \textsf{ 且 }d|\frac x c)$$ 
因此原式可以寫回
$$\sum_{c\vert x}\sum_{d|\frac x c} f(c)\mu(d)=\sum_{c|x} f(c)\sum_{d| \frac c d} \mu(d)=\sum_{c|x} f(c)[x=c]=f(x)$$
\end{proof}

或者大可以把剛剛那一段醜醜的證明忘記,改成看這個簡潔有力的證明:
\begin{proof}
$$\sum_{d|x} \mu(d)F(\frac x d)=(\mu*f*1)(x)=((\mu*1)*f)(x)=([x=1]*f)(x)=\sum_{d|x}[d=1]*f(\frac n d)=f(x)$$
\end{proof}

\problembox{The Holmes Children}{CF 776E}{
$f$ 是歐拉函數,$g(n)=\sum_{d|n} f(\frac n d)$,試著計算
$$F_k(n)=\begin{cases} f(g(n)) & k=1 \\ g(F_{k-1}(n)) & k>1,\,k\textsf{ 是偶數} \\ f(F_{k-1}(n)) & k>1,\,k\textsf{ 是奇數}\end{cases}$$
將答案模 $1000000007$ 輸出
}


\subsection{原根 Primitive Root}
\label{sec:mod:primitive}

\begin{definition}
對於質數 $p$ ,我們說 $a\in\mathbb Z_p^*$ 的\textbf{序}(order)是最小的正整數 $d$,使得$a^d=1$
\end{definition}
這樣講可能有點抽象,舉個例子:\\
$12$ 在 $\mathbb Z_{29}^*$ 的序是 $4$,因為 $12^4\equiv 1\pmod {29}$
,而且比4小的正整數次方都不是$1$\\
$5$ 在 $\mathbb Z_{101}^*$ 的序是 $25$,因為 $5^{25}\equiv 1\pmod {101}$
,而且比25小的正整數次方都不是 $1$

\begin{definition}
對於質數 $p$ ,我們說 $a\in \mathbb Z_p^*$ 是 $\mathbb Z_p^*$ 的\textbf{原根}(Primitive Root,群論用語會翻 generator),如果比 $p-1$ 小的 $a$ 的正整數冪次都不是 $1$ ,也就是說,$a$ 的序是 $p-1$。
\end{definition}
可以證明這樣的 $a$ 一定存在,證明這裡先沒寫出來。

\begin{theorem}
若 $a$ 是 $\mathbb Z_p^*$ 的原根,則 $a^0,a^1,a^2,\ldots,a^{p-2}$ 一一對應到$1,2,3,4,\ldots,p-1$
\end{theorem}
\begin{proof}
若$0\leq i<j<p-2,\,a^i=a^j$,則依據消去定理 $a^{j-i}\equiv 1\pmod p$ 不符合定義
\end{proof}
例如 $3$ 是 $\mathbb Z_7^*$ 的原根:
$$\mathbb Z^*_7=\{1,2,3,4,5,6\}$$
而 $3$ 的冪次表如下

\begin{table}[h]
	\begin{tabularx}{\textwidth}{X | X | X | X | X| X | X }
		$t$		& 1	& 2	&3  &4  &5  &6  \\ \hline
		$3^{t}$	& 3	& 2	&6  &4  &5  &1  \\
	\end{tabularx}
	\label{tab:mod:primitive:table}
	\caption{3的冪次表}
\end{table}

\begin{theorem}
正整數 $d|p-1$,$\mathbb Z_p^*$ 中序為 $d$ 的元素剛好有 $\Phi(d)$ 個
\end{theorem}
\begin{proof}
那些元素就是 $\{a^k:gcd(k,p-1)=\frac{p-1}{d}\}$ 中的元素,$a$ 是任一個原根,至於為什麼它們的序是 $d$:\\
$(a^k)^x=1\Rightarrow p-1|kx $ 再利用引理 \ref{EuclidsLemma} 可以推到
$d|x$
\end{proof}

給一個質數 $p$ ,我們有辦法找到任一個原根嗎?
我們當然可以一個個試 $2,3,4,\ldots,p-1$ ,然後將它們從 $1,\ldots,p-1$ 次方都算過一次,但若 $p\leq 10^9$,可能就沒有那麼容易了,關於這個問題,有一種機率解法是這樣:

\begin{algorithm} \small%[t] % top of the page
	\caption{找一個原根的演算法}
	\label{alg:number:primativeRoot:findingGenerator}
%	\algsetup{linenosize=\small, linenodelimiter=.}
	\begin{algorithmic}[1]
		\State 先把 $p-1$ 作因式分解,$p-1=\prod_{i=1}^r q_i^{e_i}$
		\State 對於每一種 $q_i$ 隨便挑一個數 $\alpha_i$ 使得 $\alpha_i^{\frac {p-1} {q_i}}\neq 1$ ,令 $\gamma_i=\alpha_i^{\frac {p-1}{{q_i}^{e_i}}}$ ,失敗了就再來一遍
		\State 輸出答案 $\gamma= \prod_{i=1}^r\gamma_i$ 
		\end{algorithmic}
\end{algorithm}

在每一步驟中,簡單說明一下為什麼要這樣做:
\begin{enumerate}
\item 第二步,如果這樣挑可以令 $\gamma_i^{{q_i}^{e_i-1}}\neq 1$ ,且 $\gamma_i^{q_i^{e_i}}=1$ 。而挑中的機率其實挺大的,假設 $a$ 是原根,只要挑中的不是 $a^{q_i},a^{2q_i},a^{3q_i},...,a^{p-1}=1$ 即可,平均做兩次以內就找得到。在此,$\gamma_i$ 的序是$q^{e_i}$。
\item 第三步,某個理論(群論)告訴我們:如果 $a,b$ 的序互質,則 $ab$ 的序就是各自的序的乘積,因此 $\gamma$ 的序是 $p-1$ ,講白一點,$\gamma$ 就是原根。
\end{enumerate}

2. 3. 步驟複雜度的期望值是 $\mathcal{O}(r\log p)$,$r$ 是質因數個數,已經幾乎是常數了,因此最慢的部分還是在分解 $p-1$ 

現在問題反過來,給一個數 $\gamma$ ,它是不是$\mathbb Z_p^*$ 的原根?
如果是的話,假設 $a$ 是另一個原根,則 $\gamma=a^d$,$\gcd(p-1,d)=1$,否則一定有個質數 $q$ 使得 $q|gcd(p-1,d)$,則 $p-1|\frac {p-1} q d$,我們只要一一測試分解 $p-1$ 的質數,看看$\gamma^{\frac{p-1}{q}}$ 是否等於一就好了。

以下附上程式碼,主函式分為兩個部分,第一部分是把一百萬以內的質數用篩法找出來,第二部分,對於這些質數,利用上述方法找出它們的原根,再來是測試原根的驗證程式:
\begin{lstlisting}[caption=尋找與測試原根]
#include <bits/stdc++.h>
using namespace std;
typedef unsigned long long LL;
vector<LL > Qs;

#define MAXN 1000001
LL Divider[MAXN] = {};
void sieve(LL N = MAXN) {
  for (LL p = 2; p < MAXN; ++p) {
    if (Divider[p] == 0) {
      for (LL n = p; n < MAXN; n += p) {
        Divider[n] = p;
      }
    }
  }
}
LL modPow(LL a, LL power, LL mod) {
  LL ans = 1;
  for (LL cur = a; power; power >>= 1, cur = cur * cur % mod) {
    if (power & 1) ans = ans * cur % mod;
  }
  return ans;
}
void factor(LL N) { // 把分解p-1的質數都找出來
  while (N != 1) {
    Qs.push_back(Divider[N]);
    while (Divider[N] == Qs.back())
      N /= Divider[N];
  }
}
LL root(LL prime) { // 找原根
  Qs.clear();
  factor(prime - 1);
  LL gamma = 1;
  for (LL q_i : Qs) { // Range-based for loop since C++11
    LL alpha_i = 1, b ,N = prime - 1;
    while (N % q_i == 0) N /= q_i;
    do {
      ++alpha_i; // 沒錯,這就是我的隨機函數!
      b = modPow(alpha_i, (prime - 1) / q_i, prime);
    } while(b == 1);
    gamma = gamma * modPow(alpha_i, N, prime) % prime;
  }
  return gamma;
}
// 測試 a 是否為原根
LL isPrimitiveRoot(LL a, LL prime, vector<LL > &Qs) {
  for (LL q : Qs) {
    if (modPow(a, (prime - 1) / q, prime) == 1)
      return false;
  }
  return true;
}
int main() {
  sieve();//篩法
  for (LL p = 2; p < MAXN ; ++p) {
    if (Divider[p] == p) { // p 是質數
      LL a = root(p); // 找一個原根
      if (isPrimitiveRoot(a, p, Qs))
        printf("%6lld has order %6lld under module %6lld\n",
          a, p - 1, p);
      else puts("test fail.");
    }
  }
}
\end{lstlisting}


\subsection{中國剩餘定理 Chinese Remainder Theorem}
\label{sec:mod:CRT}

\begin{theorem}[中國剩餘定理]
令 $\{n_k\}^k_{i=1}$ 為兩兩互質的正整數,令 $a_1,a_2,...,a_k$ 為任意的整數,則存在 $a\in\mathbb{Z}$ 使得滿足:$a\equiv a_i \pmod {n_i} \ ,\ i=1,\cdots ,k$
再者,令 $n={\prod_{i=1}^k n_i}$,$a^{'}\in\mathbb{Z}$ 也是一個解若且唯若 $a \equiv a^{'} \pmod n$
\end{theorem}
\begin{proof}
如果我們可以知道 $e_i\equiv\begin{cases} 1 \pmod {n_i} &  \\ 0 \pmod {n_j} &  j\neq i \end{cases}$,則可以造出 $a\equiv\sum_{i=1}^k a_ie_i\pmod n$
要如何找出 $e_i$ 呢?用歐幾里德擴充算法:

令 $n_i^*=\frac n {n_i}=\prod_{i=1,i\neq j}^kn_i$ ,有 $sn_i+tn_i^*=1$,則 $e_i:=tn_i^*$ 就完成了
\end{proof}

\begin{lstlisting}[caption=測試中國剩餘定理]
#include <bits/stdc++.h>
#define MAXN 101
using namespace std;
typedef long long LL;
typedef pair< LL , LL > ii;

ii exd_gcd( LL a , LL b ){//return s,t
  if( a % b == 0 ) return ii( 0 , 1 );
  ii T = exd_gcd( b , a % b );
  return ii( T.second , T.first - a / b * T.second );
}
LL a[ MAXN ] = { 3, 8, 6 }, n[ MAXN ] = { 17, 13, 15 }, e[ MAXN ], k = 3;

int main(){
  LL prodN = 1, ans = 0, nStar;
  for(LL i = 0 ; i < k ; ++ i ) prodN *= n[ i ];
  for(LL i = 0 ; i < k ; ++ i )
    nStar = prodN / n[i], e[i] = exd_gcd( n[ i ], nStar).second * nStar;
  for(LL i = 0 ; i < k ; ++ i ) (ans += e[ i ] * a[ i ] % prodN) %= prodN;
  printf( "%lld\n" , ( ans + prodN ) % prodN );
}
\end{lstlisting}

%\section{組合計數 Combinatorics}
%\subsection{加法與乘法原理 Addition \& Product Rule}
%\subsection{排容原理 Inclusion–exclusion Principle}
%\subsection{排列與組合 Permutation \& Combination}
%\subsection{遞迴關係式}
%\subsection{卡特蘭數}
%\subsection{玻利亞計數定理}


\section{快速傅立葉變換 Fast Fourier Transform}

\subsection{生成函數 Generating Function}
\label{sec:fft:gf}

生成函數是一種多項式的衍生函數$F\left(x\right)=a_0+a_1x+a_2x^2+a_3x^3...$,函數行為意義不大,主要用途是對於用來查他的第N次項,可以把它想像成是有無限多項的陣列。
像是$G\left(x\right)=1+\frac 1 2x+\frac 1 4x^2+\frac 1 8x^3+\frac 1 {16}x^4...$ 可以想像成$<1,\frac 1 2,\frac 1 4,\frac 1 8,\frac 1 {16},\frac 1 {32},\frac 1 {64}...>$

\begin{definition}
我們定義一個序列$<a_i>_{i=0}^\infty$的生成函數$F\left(x\right)$可被寫作$F\left(x\right)=\sum_{i=0}^{\infty}a_ix^i$
對於一個生成函數,我們用$[x^n]F\left( x\right)$ 來代表$F\left(x\right)$的第n次項
\end{definition}

相加(減)相乘就跟多項式一樣
\begin{definition}
令
$$A(x)=\sum_{i=0}^{\infty}a_ix^i,B(x)=\sum_{i=0}^{\infty}b_ix^i$$
兩個生成函數的相加定義為
$$A(x)+B(x)=\sum_{i=0}^{\infty}(a_i+b_i)x^i$$
兩個生成函數的相乘定義為
$$[x^n](AB)(x)=\sum_{i=0}^n\sum_{j=0}^na_ib_j[i+j=n]=\sum_{i=0}^n a_ib_{n-i}=\sum_{i=0}^n a_{n-i}b_i$$
\end{definition}

生成函數與dp還可以扯得上邊,甚至是拿來分析dp的好工具呢\\
如果遇到像是$G(x)=1+kx+k^2x^2+k^3x^3+k^4x^4...$的生成函數,令$A(x)=\sum_{i=0}^{\infty}a_ix^i$,計算$[x^n](GA)( x)$時,不必花$\mathcal{O}(n^2)$將他們相乘,留意到
$$[x^n]\left(GA\right)\left( x\right)=\sum_{i=0}^{n}k^{n-i}a_{i}=a_n+k\sum_{i=0}^{n-1}k^{n-1-i}a_i=a_n+k[x^{n-1}]\left(GA\right)\left( x\right)$$變成了漂亮的dp式。

\problembox{01背包計數問題}{經典問題}{
有一個小偷有容量$V$的背包,考慮$N$個物品,每個物品的體積不同,但我們今天不問小偷能不能塞滿這個背包,我們想知道它有多少種方法把這個背包塞滿,請在$\mathcal{O}(NV)$時間內對於每種$V$都輸出該答案,如果今天方法A所塞的物品,方法B都有了,而且方法B所塞的物品,方法A都有了,那我們就說方法A跟方法B是一樣的,不然這兩種方法就是不一樣的
}
令第$i$個物品重量為$w_i$,那答案就是
$[x^V]\prod_{i=0}^{N-1}(1+x^{w_i})$
這其實就是dp式的表現,令\lstinline{dp[i][v]}為前i個選項中,容量為v的答案:
則\lstinline{dp[i][v]=dp[i-1][v-w_i]+dp[i-1][v]}
對應到的生成函數是
$$\prod_{j=0}^{i-1}(1+x^{w_j})\times (1+x^{w_i})$$
\begin{theorem}
一個生成函數函數乘上$(1+x+x^2+x^3+x^4+...)$,會變成它本身的前綴和
\end{theorem}

考慮到幾何級數,有時候我們會把生成函數$(1+x+x^2+x^3+x^4+...)$寫成$\frac 1 {1-x}$(因為在$x$絕對值小於一時兩邊是一樣的)
想要把 $I$ 個區間都加$k_i$,而題目只有在最後做query,利用生成函數的思想,不必用線段樹就可以把複雜度做到$O\left(N + I \right)$,簡單又好寫:
留意到

$$        (x^n+x^{n+1}+x^{n+2}+...x^{m})=(1+x+x^2+x^3+x^4+...)(x^n-x^{m+1})$$
$$k_1(x^{n_1}+x^{{n_1}+1}+...x^{m_1})+k_2(x^{n_2}+x^{{n_2}+1}+...x^{m_2})+...$$
   $$ :=\sum^I_{i=1}k_i\sum^{m_i}_{j={n_i}}x^j=\sum^I_{i=1}k_i\frac{(x^n-x^{m+1})}{1-x}=\frac{1}{1-x}\sum^I_{i=1}k_i(x^n-x^{m+1})$$

告訴我們可以先在陣列$n_i,m_{i+1}$個別加上 $k_i,-k_i$ ,最後再做前綴和即得到答案

\problembox{二項式恆等式}{證明題}{
利用生成函數證明:
$$\sum_{i=0}^r {n\choose i}{m \choose{r-i}} = {{m+n}\choose{r}}$$
}

那為什麼要提到生成函數呢,它的功用主要在於計數分析,不過有一部分計數的問題需要用到生成函數乘積,這時候一個好的多項式乘法algo會變得相當重要,因為要算$n,m$次多項式$P(x),Q(x)$相乘的時候,如果沒有特別好的條件的話,生成函數相乘硬解的複雜度會是$\mathcal{O}(n^2)$,但假設現在我們知道了$\{(x_i,P(x_i)Q(x_i))\}_{i=1}^{n+m+1}$,利用等會兒提到的多項式插值唯一定理,可以找回$P(x)Q(x)$


\newcommand{\pie}[1]{%
\begin{tikzpicture}
 \draw (0,0) circle (1ex);\fill (1ex,0) arc (0:#1:1ex) -- (0,0) -- cycle;
\end{tikzpicture}%
}


\subsection{離散傅立葉變換 Discrete Fourier Transform}
\label{sec:fft:dft}

在進入離散傅立葉變換(DFT)之前,先看個引理吧

\begin{lemma}[多項式插值唯一定理 Uniqueness of an interpolating polynomial]
一個平面上一堆點$\{(x_i,y_i)\}^{n-1}_{i=0}$, $x_i$ 們各不相同,存在唯一一個多項式$P(x)$使得$P(x_i)=y_i$,使得多項式最高次項小於$n$
\end{lemma}
\begin{proof}
假設我們想要知道的多項式是$P(z)=a_0+a_1z+a_2z^2+...+a_{n-1}z^{n-1}$
問題就變成了
\begin{enumerate}
\item $a_0,a_1,...,a_{n-1}$ 是什麼
\item 解為什麼是唯一的
\end{enumerate}


我們有恆等式:
\[
\left(		
\begin{array}{ccccc}		
   1 & x_0 & x_0^2 & \dots & x_0^{n-1}  \\		
   1 & x_1 &  x_1^2 & \dots & x_1^{n-1}  \\	
   1 & x_2 &  x_2^2 & \dots & x_2^{n-1}  \\	
   \vdots & \vdots & \vdots & \vdots & \vdots \\	
   1 & x_{n-1} & x_{n-1}^2 & \dots & x_{n-1}^{n-1} \\		
    \end{array}		
\right)		
\left(		
\begin{array}{c}		
  a_0 \\		
  a_1 \\
  a_2 \\
  \vdots \\
  a_{n-1}		
 \end{array}		
\right)		
 =		
\left(		
\begin{array}{c}		
  P(x_0)\\		
 P(x_1) \\
  P(x_2) \\
  \vdots \\
  P(x_{n-1})		
 \end{array}		
\right)	
 =		
\left(		
\begin{array}{c}		
  y_0 \\		
  y_1 \\
  y_2 \\
  \vdots \\
  y_{n-1}		
 \end{array}		
\right)
 \]
 

左邊那個矩陣,我們叫它范德蒙矩陣(Vandermonde matrix)
注意到上式中, $y_i,x_i$ 我們已經有了,看來我們還需要擅長取反矩陣找 $a_i$ 的朋友呢

等等,這矩陣是可逆的嗎?

可以配合歸納法來證明:
\[
\textsf{det}
\left(		
\begin{array}{ccccc}		
   1 & x_0 & x_0^2 & \dots & x_0^{n-1}  \\		
   1 & x_1 &  x_1^2 & \dots & x_1^{n-1}  \\	
   1 & x_2 &  x_2^2 & \dots & x_2^{n-1}  \\	
   \vdots & \vdots & \vdots & \vdots & \vdots \\	
   1 & x_{n-1} & x_{n-1}^2 & \dots & x_{n-1}^{n-1} \\		
    \end{array}		
\right)		
=
\prod_{0\leq i<j<n}(x_j-x_i)
 \]

因為 $x_i$ 們是各不相同,這個矩陣的行列式 (det) 不為零,當然是可逆的,而且這同時也告訴我們$a_0,a_1,...,a_{n-1}$是唯一決定的!
\end{proof}

高斯消去法取反矩陣要花$\mathcal{O}(n^3)$,拉格朗日插值法要花$\mathcal{O}(n^2)$,然而,如果 $x_i$ 選得很特別,有辦法做到$\mathcal{O}(n\lg n)$時間在 $a_i,y_i$ 間快速轉換,這個我們叫它FFT,在介紹FFT以前,先從DFT來下手。


假設有一種數字 $x$ ,對於 $n$ 它滿足 $\sum_{i=0}^{n-1}x^i=0$ ,除了零以外真的有這種鬼數字存在嗎?有的!\\
考慮對於$nd+1$的質數$p$,討論$\mathbb Z_p^*$時,這種鬼數字會存在,或者是不真實(real)而複雜(complex)的數,我們叫它複數$\mathbb{C}$。

先說第一種情況:
\begin{theorem}
\label{prime}
若$p$是一個$nd+1$的質數,$a$是一個原根,令$b=a^d$,則$b$的序為$n$
\end{theorem}
\begin{proof}
令$x$為$b$的序\\
若$x$比$n$小,則$b^x=a^{xd}=1$,則$a$不是原根,因為它的序比$nd$小,矛盾。\\
而$b^n=a^{nd}=a^{p-1}=1$表示$b$的序比$n$小或等於$n$,因此$b$的序是$n$
\end{proof}
在此 $\sum_{i=0}^{n-1}x^i=0$ 的非零解是什麼呢?答案是在模p下的 $b^k, k = 1,2,...n-1$\\
為了標記方便,對於$\mathbb{Z}_p^*$的元素,我們在這章節統稱共軛為反元素,以$\overline{b}=b^{-1}$表示。

對於第二種情形,或許已經有人知道複數的概念,但這裡再提一下,複數是一個平面$\mathbb{C}=\mathbb R+i\mathbb R$,其中 $i\times i = -1$,複數支援加減乘除,若$A=a_1+ia_2,B=b_1+ib_2$,則
\begin{enumerate}
\item 我們稱$A$的共軛複數為 $\overline{A}=a_1-ia_2$
\item $A+B=(a_1+b_1)+i(a_2+b_2)$
\item $A-B=(a_1-b_1)+i(a_2-b_2)$
\item $A\times B=(a_1b_1-a_2b_2)+i(a_1b_2+a_2b_1)$
\item $A\div B=\frac{a_1+ia_2}{b_1+ib_2}=\frac {(a_1+ia_2)(b_1-ib_2)}{(b_1-ib_2)(b_1+ib_2)}=\frac {(a_1b_1+a_2b_2)+i(a_2b_1-a_1b_2)}{b_1^2+b_2^2}$
\end{enumerate}

物件 \lstinline{std::complex} 支援這些運算,包括三角函數的使用。

可以用xy座標畫出複數的點,一個複數的絕對值為它到原點 $0+i0$ 的歐幾里德距離(用尺量出來的那個距離)

這些數字都不是自然存在的數,你的午餐價格不可能是 $3+2i$,但就是因為複數有很好用的結構(複結構),才會被發展出來做解方程式的根、FFT之類的事

在此 $\sum_{i=0}^{n-1}x^i=0$ 的非零解是什麼呢?答案是 $e^{\frac{2\pi i k}{n}}, k = 1,2,...n-1$,這些數的共軛跟反元素是一樣的\\
我們有歐拉公式 Euler Formula:
$e^{ix}=\cos x + i \sin x$\\
這是展開式 $e^x=\frac 1 {0!}+\frac x {1!}+\frac {x^2} {2!}+\frac {x^3} {3!}+...$  在複平面上推廣所得出的結果。

如果是$\mathbb Z_p^*$我們令$w^k_N=b^k$,如果是$\mathbb{C}$上我們令 $w^k_N=e^{2\pi i\frac{ k}{N}}$ (說穿了就只是將 $1$ 對著原點逆時針旋轉 $\frac k N$ 個圓周的複數,如果 $k<0$ 就代表順時針旋轉 $\frac {|k|} N$ 個圓周),為了直觀我們就用圓餅圖示範(下表),這裡我們叫它「派運算」,派運算表示時為了方便起見$N$都是8,值得一提的是,派運算不管是對於$\mathbb{Z}_p^*$上的$w_N=b$,還是$\mathbb{C}$上的$w_N=e^{\frac{ 2\pi i}{N}}$,下面敘述都是滿足的,因此講者以下不再區分兩者的差別:

\begin{table}[h]
	\begin{tabularx}{\textwidth}{X | X | X | X | X|X | X | X }
		$w^0_8$	& $w^1_8$	& $w^2_8$	&$w^3_8$  &$w^4_8$  &$w^5_8$  &$w^6_8$  &$w^7_8$ \\ \hline
		$\pie{0}$或$\pie{360}$	& $\pie{45}$	& $\pie{90}$	 &$\pie{135}$	 &$\pie{180}$	  &$\pie{225}$	  &$\pie{270}$	  &$\pie{315}$\\ %\hline
	\end{tabularx}
	\label{tab:fft:dft:table}
	\caption{派-複數根對照表}
\end{table}


\begin{theorem}
$w_N^i$與派運算有以下規則\\
\begin{enumerate}
\item $w^i_N\times w^j_N=w^{i+j}_N$,在派運算下就是將面積相加,例如$\pie{45} \pie{90} = \pie{135}$
\item $(w^i_N)^j=w^{ij}_N$,在派運算下就是將面積相乘上一個常數,例如$\pie{45}^3=\pie{135}$
\item $w^{-i}_N$ 是 $w^i_N$的共軛,在派運算下就是將一塊完整的派拿走原來派的面積,例如$\overline{\pie{90}}=\pie{270}$
\item $w^i_N=w^{i+N}_N$,在派運算下就是將面積模一塊完整的派,例如$\pie{45}\pie{360}\pie{360}=\pie{45}$
\item $w^{ik}_{Nk}=w^{i}_N$,在派運算下就是指$\pie{180}$切成四塊或是八塊面積都是一樣的:$\pie{180}=\pie{45}\pie{45}\pie{45}\pie{45}=\pie{90}\pie{90}$
\item $\sum_{k=0}^{N-1}(w^i_N)^k=N[N|i]$,在派運算下就像是$\sum_{k=0}^{N-1}(\pie{45})^k=0$
\end{enumerate}
\end{theorem}
\begin{proof}
我們證最後一點的派式子,若$i\neq 0\pmod N$,可以發現:
\begin{equation}
  \begin{aligned}
(\pie{360}-\pie{45})(\pie{0}+\pie{45}+\pie{90}+\pie{135}+\pie{180}+\pie{225}+\pie{270}+\pie{315})\\
=(\pie{0}+\pie{45}+\pie{90}+\pie{135}+\pie{180}+\pie{225}+\pie{270}+\pie{315})-(\pie{45}+\pie{90}+\pie{135}+\pie{180}+\pie{225}+\pie{270}+\pie{315}+\pie{360})\\
=\pie{0}-\pie{360} = 0
  \end{aligned}\nonumber
\end{equation}
因此$(\pie{0}+\pie{45}+\pie{90}+\pie{135}+\pie{180}+\pie{225}+\pie{270}+\pie{315})=\frac{\pie{0}-\pie{360}}{\pie{360}-\pie{45}}=0$
\end{proof}

注意$3\pie{90}\neq \pie{270}$而是應該要像向量一樣直接寫成$3\pie{90}$

\begin{definition}
$w^0_N,w^1_N,...w^{N-1}_N$所構成的范德蒙矩陣,我們叫它DFT 矩陣,以$D_N$簡寫
\end{definition}



\[
D_N
=
\begin{pmatrix}    
1 & w_N^0 & w_N^{0*2} & \cdots  & w_N^{0*(N-1)} \\     
1 & w_N^1 & w_N^{1*2} & \cdots  & w_N^{1*(N-1)} \\     
\vdots & \vdots & \vdots & \ddots & \vdots \\     
1 & w^{N-1}_{N} & w^{(N-1)*2}_{N} & \cdots  & w^{(N-1)*(N-1)}_{N} \end{pmatrix}
=
\begin{pmatrix}    
\pie{0} & \pie{0} & \pie{0} & \cdots  & \pie{0} \\     
\pie{0} & \pie{45} & \pie{90} & \cdots  & \pie{315)} \\     
\vdots & \vdots & \vdots & \ddots & \vdots \\     
\pie{0} & \pie{315} & \pie{270} & \cdots  & \pie{45} 
\end{pmatrix}
 \]

是個漂亮的對稱矩陣呢!

它的反矩陣長這樣:
\[
D_N^{-1}
=
\frac 1 N
\begin{pmatrix}     
1 & w_N^{-0} & w_N^{-0*2} & \cdots  & w_N^{-0*(N-1)} \\     
1 & w_N^{-1} & w_N^{-1*2} & \cdots  & w_N^{-1*(N-1)} \\     
\vdots & \vdots & \vdots & \ddots & \vdots \\     
1 & w^{1-N}_{N} & w^{(1-N)*2}_{N} & \cdots  & w^{(1-N)*(N-1)}_{N} \end{pmatrix}
=
\frac 1 N
\begin{pmatrix}     
\pie{360} & \pie{360} & \pie{360} & \cdots  & \pie{360)} \\     
\pie{360} & \pie{315} & \pie{270} & \cdots  & \pie{45} \\     
\vdots & \vdots & \vdots & \ddots & \vdots \\     
\pie{360}& \pie{45} & \pie{90} & \cdots  & \pie{315} 
\end{pmatrix}
\]

簡單來說 $D_n$ 的反矩陣是 $D_n$ 每個項取共軛再除以$N$,如果學員學過(或未來會學到)線性代數,就可以知道這是一個Unitary Matrix(忽略常數),自己乘乘看是不是單位矩陣吧。


從上述的定理我們可以知道一個小於$n$次的多項式的呈現,除了可以用$P(z)=a_0+a_1z+a_2z^2+...+a_{n-1}z^{n-1}$表示,還可以用$n$個點來代表一個唯一的多項式,這個表示法叫做點值表示法(Point Value Representation)

發現了嗎,如果有兩個多項式的點值表示法,而它們選用的那些 $x_i$ 都一樣,要怎麼得到兩個多項式相乘的點值式呢?直接把每個y座標都相乘就好了對吧。

\begin{algorithm} \small%[t] % top of the page
	\caption{離散傅立葉變換}
	\label{alg:FFT:DFT}
%	\algsetup{linenosize=\small, linenodelimiter=.}
	\begin{algorithmic}[1]
		\State 先用$D_N$矩陣乘積將 $P(\pie{0}),P(\pie{45}),P(\pie{90}),...,P(\pie{315})$,$Q(\pie{0}),Q(\pie{45}),Q(\pie{90}),...,Q(\pie{315})$算出來。
		\State 對於每個 $\pie{45}^i$ 直接相乘兩個點值表示法的多項式 $P(\pie{0})Q(\pie{0})$,$P(\pie{45})Q(\pie{45})$,...,$P(\pie{315})Q(\pie{315})$
		\State  用反矩陣$D_N^{-1}$將將 $P(x)Q(x)$ 的 $x^i$ 每一項算出
		\end{algorithmic}
\end{algorithm}


這個算法是$\mathcal{O}(n^2)$,門檻在1. 3. 步驟,但待會兒FFT我們會利用一點小技巧加速到$\mathcal{O}(n\lg n)$,這裡注意 $P(x) Q(x)$ 的次數最高只能是 $N-1$,否則會得到唯一一個多項式 $H(x)$ 使得 $H(\pie{45}^i)=P(\pie{45}^i)Q(\pie{45}^i)$ 且次數小於$N$的多項式。

為了說明DFT的正確性,以下程式碼比較DFT(on $\mathbb{C}$)跟直接做多項式相乘的差別,感受一下DFT的正確性,注意目前兩者都是$\mathcal{O}(n^2)$。

\begin{lstlisting}
using namespace std;
typedef complex<double> cpx;
#define MAXN 1000
#define PI acos(-1)
#define I cpx(0,1)

cpx D_N[ MAXN ][ MAXN ], D_N_inv[ MAXN ][ MAXN ];
cpx P[ MAXN ],Q[ MAXN ];
cpx pointValueP[ MAXN ], pointValueQ[ MAXN ];
cpx dotProduct[ MAXN ], DFTAns[ MAXN ], directlyMultiplyAns[ MAXN ];

void matrixMultiply(cpx Matrix[ MAXN ][ MAXN ],cpx *Value, cpx *Ans, int n){
  for(int i = 0 ; i < n ; ++ i ){
    Ans[ i ] = 0 ;
    for(int j = 0; j < n ; ++ j ){
      Ans[ i ] += Matrix[ i ][ j ] * Value[ j ];
    }
  }
}
void directlyPolynomialMultiply(cpx *polyP, cpx *polyQ, cpx *Ans, int n){
  for(int i = 0 ; i < n ; ++ i ){
    Ans[ i ] = 0 ;
    for(int j = 0 ; j <= i ; ++ j)
      Ans[ i ] += polyP[ j ] * polyQ[ i - j ];
  }
}
void matrixInitialization(cpx D_N[MAXN][MAXN],cpx D_N_inv[MAXN][MAXN],int finalDegree){
  for(int i = 0 ; i < finalDegree ; ++ i){//D_N矩陣與反矩陣
    for(int j = 0; j < finalDegree; ++ j){
      D_N[ i ][ j ] = exp( PI * 2 * ( i * j ) / finalDegree * I );
      D_N_inv[ i ][ j ] = ( cpx ) ( 1. / finalDegree ) * conj( D_N[ i ][ j ] );
    }
  }
}
void DFT(cpx P[MAXN],cpx Q[MAXN],cpx DFTAns[MAXN],int finalDegree){
  matrixMultiply( D_N, P, pointValueP, finalDegree);
  matrixMultiply( D_N, Q, pointValueQ, finalDegree);
  for(int i = 0; i < finalDegree ; ++ i)
    dotProduct[ i ] = pointValueP[ i ] * pointValueQ[ i ];
  matrixMultiply( D_N_inv, dotProduct, DFTAns, finalDegree);
}
int main(){
  const int PolynomialMaxValue = 50, maxDegree = 500, finalDegree = maxDegree * 2;
  for(int i = 0 ; i < maxDegree ; ++ i){//隨機給予兩個多項式初始值
    P[ i ] = rand() % PolynomialMaxValue,
    Q[ i ] = rand() % PolynomialMaxValue;
  }
  matrixInitialization( D_N , D_N_inv , finalDegree);
  
  DFT(P,Q,DFTAns,finalDegree);
  
  directlyPolynomialMultiply( P, Q, directlyMultiplyAns, finalDegree);
  for(int i = 0; i < finalDegree ; ++ i ){
    if( abs( DFTAns[ i ] - directlyMultiplyAns[ i ] ) > 1e-5 )
      puts("test fail");
  }
}
\end{lstlisting}
請讀者自行練習寫出$\mathbb{Z}_p^*$上的DFT
\subsection{快速傅立葉變換 Fast Fourier Transformation}
\label{sec:comb:gen}
「快速傅立葉變換」聽起來真是嚇死人了,好像很難的樣子,可以叫他「利用奇怪矩陣的特性所做的加速來算特定幾個點的值之演算法」,講白了就是比較快的DFT,利用DFT中,$D_N$的特性所做的加速,它的輸入輸出跟DFT沒什麼差別,除了:

1. $N$ 一定要是小的質數的乘積,這裡我們指 $2$ 的次方\\
2. 比較快,DFT是$\mathcal{O}(n^2)$,FFT只要$\mathcal{O}(n\lg n)$

為了方便說明,我們這裡一樣舉$N=8$為例

令$P(x)=a_0+a_1x+a_2x^2+a_3x^3+a_4x^4+a_5x^5+a_6x^6+a_7x^7$
要算出
\[
\begin{pmatrix}    
P(\pie{0})\\
P(\pie{45})\\
P(\pie{90})\\
P(\pie{135})\\
P(\pie{180})\\
P(\pie{225})\\
P(\pie{270})\\
P(\pie{315})
\end{pmatrix} 
=
\begin{pmatrix}    
\pie{0} & \pie{0} & \pie{0} & \pie{0} & \pie{0} & \pie{0} &\pie{0} & \pie{0}\\     
\pie{0} & \pie{45} & \pie{90} & \pie{135} & \pie{180} & \pie{225} &\pie{270} & \pie{315}\\     
\pie{0} & \pie{90} & \pie{180} & \pie{270} &\pie{0} & \pie{90} & \pie{180} & \pie{270}\\     
\pie{0} & \pie{135} & \pie{270} & \pie{45} & \pie{180} & \pie{315} &\pie{90} & \pie{225}\\    
\pie{0} & \pie{180} & \pie{0} & \pie{180} & \pie{0} & \pie{180} &\pie{0} & \pie{180}\\    
\pie{0} & \pie{225} & \pie{90} & \pie{315} & \pie{180} & \pie{45} &\pie{270} & \pie{135}\\
\pie{0} & \pie{270} & \pie{180} & \pie{90} & \pie{0} & \pie{270} &\pie{180} & \pie{90}\\     
\pie{0} & \pie{315} & \pie{270} & \pie{225} & \pie{180} & \pie{135} &\pie{90} & \pie{45}  
\end{pmatrix}
\begin{pmatrix}    
a_0\\
a_1\\
a_2\\
a_3\\
a_4\\
a_5\\
a_6\\
a_7
\end{pmatrix}
\]



需要這些資訊:\\
令
$$P^{[0]}(x)=a_0+a_2x+a_4x^2+a_6x^3$$
$$P^{[1]}(x)=a_1+a_3x+a_5x^2+a_7x^3$$

則
$$P(x)=P^{[0]}(x^2)+x\cdot P^{[1]}(x^2)$$

看起來我們總共要算出
$$P^{[0]}(\pie{0}^2),P^{[0]}(\pie{45}^2),P^{[0]}(\pie{90}^2),P^{[0]}(\pie{135}^2),P^{[0]}(\pie{180}^2),P^{[0]}(\pie{225}^2),P^{[0]}(\pie{270}^2),P^{[0]}(\pie{315}^2)$$
$$P^{[1]}(\pie{0}^2),P^{[1]}(\pie{45}^2),P^{[1]}(\pie{90}^2),P^{[1]}(\pie{135}^2),P^{[1]}(\pie{180}^2),P^{[1]}(\pie{225}^2),P^{[1]}(\pie{270}^2),P^{[1]}(\pie{315}^2)$$

把平方乘進去發現需要的只有
$$P^{[0]}(\pie{0}),P^{[0]}(\pie{90}),P^{[0]}(\pie{180}),P^{[0]}(\pie{270})$$
$$P^{[1]}(\pie{0}),P^{[1]}(\pie{90}),P^{[1]}(\pie{180}),P^{[1]}(\pie{270})$$

意思是說我們只要算出

\[
\begin{pmatrix}    
P^{[0]}(\pie{0})\\
P^{[0]}(\pie{90})\\
P^{[0]}(\pie{180})\\
P^{[0]}(\pie{270})
\end{pmatrix} 
=
\begin{pmatrix}    
\pie{0}  & \pie{0} & \pie{0}  &\pie{0} \\     
\pie{0}  & \pie{90} & \pie{180} &\pie{270} \\     
\pie{0} & \pie{180} & \pie{0} & \pie{180} \\  
\pie{0} & \pie{270} & \pie{180} & \pie{90} \\     
\end{pmatrix}
\begin{pmatrix}    
a_0\\
a_2\\
a_4\\
a_6
\end{pmatrix}
\]

\[
\begin{pmatrix}    
P^{[1]}(\pie{0})\\
P^{[1]}(\pie{90})\\
P^{[1]}(\pie{180})\\
P^{[1]}(\pie{270})
\end{pmatrix} 
=
\begin{pmatrix}    
\pie{0}  & \pie{0} & \pie{0}  &\pie{0} \\     
\pie{0}  & \pie{90} & \pie{180} &\pie{270} \\     
\pie{0} & \pie{180} & \pie{0} & \pie{180} \\  
\pie{0} & \pie{270} & \pie{180} & \pie{90} \\     
\end{pmatrix}
\begin{pmatrix}    
a_1\\
a_3\\
a_5\\
a_7
\end{pmatrix}
\]
便可以在$\mathcal{O}(n)$時間內求出原來的式子。

瞧!發生了什麼事?\\
只要把一個FFT問題分解成兩個小的FFT問題,需要的運算量一瞬間少了一半呢!\\
當問題變得最小時,要解的矩陣像這樣
\[
\begin{pmatrix}    
\pie{0} \\   
\end{pmatrix}
\begin{pmatrix}    
a_0\\
\end{pmatrix}
=
\begin{pmatrix}    
a_0\\
\end{pmatrix}
\]
寫成分析式是這樣:
$$T(N)=2T(\frac N 2)+\Theta(N)$$
哇~!原來FFT就跟MergeSort一樣簡單呢!複雜度分析出來是$\Theta(N\lg N)$,這真的是太棒了!!!
乘上$D_N^{-1}$時,注意到$D_N^{-1}=\frac{1}{N} \overline{D_N}$,因此改一個小小的根,然後結果再乘上$\frac{1}{N}$就好了

下列程式碼比較使用FFT算跟剛剛三個步驟程式碼看起來像這樣:
\begin{lstlisting}[caption=FFT:快速乘DFT矩陣的實現]
using namespace std;
typedef complex<double> cpx;
#define MAXN (1 << 15)
#define PI acos(-1)
#define I cpx(0, 1)
#define EPS 1e-4

cpx dotProduct[MAXN], DFTAns[MAXN];
vector<cpx> P(MAXN), Q(MAXN), pointValueP(MAXN), pointValueQ(MAXN);
vector<cpx> FFTAns(MAXN), directlyMultiplyAns(MAXN);

vector<cpx> directlyPolynomialMultiply(
  const vector<cpx> &PolyP,
  const vector<cpx> &PolyQ,
  int n
) {
  vector<cpx> ret( n);
  for (int i = 0; i < n; ++i) {
    ret[i] = 0;
    for(int j = 0; j <= i; ++j)
      ret[i] += PolyP[j] * PolyQ[i - j];
  }
  return ret;
}

vector<cpx> FFT(const vector<cpx> &P, int n, cpx root) {
  if (n == 1) return P
  const int half_n = n/2;
  vector<cpx> ret(n, 0), oddP(half_n, 0), evenP(half_n, 0);
  for (int i = 0; i < half_n; ++i)
    evenP[i] = P[2 * i],
    oddP[i]  = P[2 * i + 1];
  evenP = FFT(evenP, half_n, root * root);
  oddP = FFT(oddP, half_n, root * root);
  cpx base = 1;
  for (int i = 0; i < n; ++i) {
    ret[i] = evenP[i % half_n] + oddP[i % half_n] * base;
    base *= root;
  }
  return ret;
}

int main() {
  const int PolynomialMaxValue = 50, maxDegree = 512, finalDegree = maxDegree * 2;
  // 隨機給予兩個多項式初始值
  for (int i = 0; i < maxDegree; ++i)
    P[i] = rand() % PolynomialMaxValue,
    Q[i] = rand() % PolynomialMaxValue;

  directlyMultiplyAns = directlyPolynomialMultiply(P, Q, finalDegree);

  P = FFT(P , MAXN , exp(2*PI/MAXN*I));
  Q = FFT(Q , MAXN , exp(2*PI/MAXN*I));
  for (int i = 0; i < MAXN; ++i)
    FFTAns[i] = P[i] * Q[i] * (1./MAXN);
  FFTAns = FFT(FFTAns, MAXN, exp(-2*PI/MAXN*I));

  for (int i = 0; i < finalDegree; ++i) {
    if (abs(FFTAns[i] - directlyMultiplyAns[i]) > EPS)
      puts("test fail");
  }
}
\end{lstlisting}

\problembox{將常數變小}{經典問題}{

1. 排序問題有個技巧是:當問題夠小的時候,直接使用$\mathcal{O}(n^2)$的插入排序,因為問題夠小時(一般來說陣列長度小於30),插入排序的常數表現非常優。試試看在矩陣夠小時,直接乘矩陣會稍微快一些。

2. 將FFT寫成迭代的,請讀者回去查詢使用bitReverse的迭代FFT。

}

\problembox{Golf Bot}{UVa 12879}{
非負整數集$G=\{0,a_1,a_2,a_3...,a_n\},a_n\leq 2\times10^5$,可以從$G$選兩個元素(可重複挑選同一種元素),請問有沒有辦法選出兩個元素使得它們的和為$m$?對於各種指定的$m$請輸出有沒有辦法達成
}

這題可以用\lstinline{std::bitset}寫$\mathcal{O}(n^2)$的算法,但這題若改成。

\problembox{Golf Bot 改}{UVa 12879改}{
非負整數集$G=\{0,a_1,a_2,a_3...,a_n\},a_n\leq 2\times10^5$,可以從$G$選兩個元素(可重複挑選同一種元素),請問有沒有辦法選出兩個元素使得它們的和為$m$?對於各種指定的$m$請輸出有幾種辦法達成
}
就老實地使用FFT吧!


下面有一題可以使用FFT以及快速冪就可以算出來的題目
\problembox{Thief in a Shop}{CF 632E}{
一個小偷要從商店偷剛好$k$個物品,有$N$種物品,每種物品都有無限多個,第$i$種物品價值$a_i$元,問他離開商店時,背包裡的物品價值可能是多少?
}


\newpage
